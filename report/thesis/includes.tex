%%%%%%%%%%%%%%%%%%%%%%%%%%%%%%%%%%%%%%%%%
% University Assignment Title Page 
% LaTeX Template
% Version 1.0 (27/12/12)
%
% This template has been downloaded from:
% http://www.LaTeXTemplates.com
%
% Original author:
% WikiBooks (http://en.wikibooks.org/wiki/LaTeX/Title_Creation)
%
% License:
% CC BY-NC-SA 3.0 (http://creativecommons.org/licenses/by-nc-sa/3.0/)
% 
%
%%%%%%%%%%%%%%%%%%%%%%%%%%%%%%%%%%%%%%%%%
%----------------------------------------------------------------------------------------
%	PACKAGES AND OTHER DOCUMENT CONFIGURATIONS
%----------------------------------------------------------------------------------------
\usepackage[a4paper,top=3cm,bottom=2cm,left=2cm,right=2cm,marginparwidth=1.75cm]{geometry}
\usepackage{textpos}
\usepackage{natbib} % for bibliography
% \usepackage[square]{natbib} % for bibliography
\usepackage{tabularx,longtable,multirow,subfigure,caption}%hangcaption
\usepackage{fncylab} %formatting of labels
\usepackage{fancyhdr} % page layout
\usepackage{url} % URLs
\usepackage[english]{babel}
\usepackage{amsmath}
% \usepackage{txfonts}
% \usepackage{smaller}
\usepackage{changepage}
\usepackage{graphicx}
\usepackage{dsfont}
\usepackage{epstopdf} % automatically replace .eps with .pdf in graphics
\usepackage{backref} % needed for citations
\usepackage{array}
\usepackage{latexsym}
\usepackage[pdftex,pagebackref,hypertexnames=false,colorlinks=true,allcolors=blue]{hyperref} % provide links in pdf
\usepackage{amsfonts}
\usepackage{booktabs}
\usepackage{nth}
\usepackage{todonotes}
\usepackage{placeins}
\usepackage{scalerel}
\usepackage{amssymb}
\usepackage{relsize}
\usepackage{graphicx}
\graphicspath{ {./figures/} }
\usepackage{tikz,lipsum,lmodern}
\usepackage[most]{tcolorbox}
\usepackage{xcolor}
\tcbuselibrary{theorems, breakable, skins, raster}
\newtcbtheorem[number within=section]{Definition}{Def.}%
    {
    enhanced, % tcolorbox styles
    %     %attach boxed title to top center={yshift=-2.5mm},
        % colback=white, colframe=black, colbacktitle=white, coltitle=black,
    %     boxed title style={size=small,colframe=white},
        % fonttitle=\bfseries,
    %     rounded corners=all,
    %     toptitle=1ex, top=0.5ex, % a little extra space at top, a little less before content
    %     titlerule=-1ex, % get rid of separator rule
    %     title={#1},
    %     fontupper=\itshape, % make theorem content italics
    %     description delimiters parenthesis, % parentheses around theorem title
    %     description font=\normalfont,% no bold for theorem title
    %     separator sign none,% no punctuation after theorem name
    %     breakable
    }{th}
\newtcbtheorem[number within=section]{Property}{Prop.}%
    {
    enhanced,
    lowerbox=ignored,
    }{th}

\makeatletter
\providecommand{\leftsquigarrow}{%
  \mathrel{\mathpalette\reflect@squig\relax}%
}
\newcommand{\reflect@squig}[2]{%
  \reflectbox{$\m@th#1\rightsquigarrow$}%
}
\makeatother

% Aeneas imports
\newif\iflong
\newif\ifshort
\shortfalse
\longtrue

\usepackage[utf8]{inputenc}
\usepackage[T1]{fontenc}
\usepackage{xspace}
\usepackage{xparse}
  \NewDocumentCommand{\li}{v}{\textbf{\footnotesize\texttt{#1}}}
\usepackage{minted}
\usepackage{adjustbox}
% \setminted{fontsize=\footnotesize,linenos}
\setminted{linenos}
\usepackage{wrapfig}
\usepackage{tikz}
  \usetikzlibrary{calc}
  \usetikzlibrary{shadows,graphs}
\usepackage{amsmath}
\usepackage{mathpartir}
  \renewcommand{\DefTirName}[1]{\hypertarget{#1}{\TirName {#1}}}
  \newcommand{\Rule}[1]{\hyperlink{#1}{\TirName {#1}}}

\newcommand{\sslash}{\mathbin{/\mkern-6mu/}}
\newcommand{\ocomment}[1]{\ensuremath{\textcolor{gray}{\sslash\textit{ #1}}}}
\newcommand{\grayit}[1]{\textcolor{gray}{\textit{#1}}}
\newcommand{\white}[1]{\textcolor{white}{#1}}
% \newcommand{\codeexample}[2]{
%   \begin{minted}[linenos, numbersep=-5pt]{rust}
%     #1
%   \end{minted}

%   \textcolor{gray}{\textit{#2}}
% }

\newcommand{\sref}[1]{Section~\ref{sec:#1}}
\newcommand{\fref}[1]{Figure~\ref{fig:#1}}
\newcommand{\tref}[1]{Table~\ref{table:#1}}

\newcommand{\aeneas}{\textsc{Aeneas}\xspace}
\newcommand{\charon}{\textsc{Charon}\xspace}
\newcommand{\fstar}{F$^\ast$\xspace}
\newcommand{\betree}{B$^\varepsilon$tree\xspace}

\DeclareRobustCommand{\son}[1]{ {\begingroup\color{red!60!black}{(Son) #1}\endgroup} }
\DeclareRobustCommand{\jonathan}[1]{ {\begingroup\color{teal}{(Jonathan) #1}\endgroup} }
\DeclareRobustCommand{\aymeric}[1]\relax

\newcommand\kw[1]{\ensuremath{\mathsf{#1}}}
\newcommand\tbrw[2]{\ensuremath{\mathsf{\&}^#1\,#2}}
\newcommand\tmbrw[2]{\ensuremath{\mathsf{\&}^#1\mathsf{mut}\,#2}}
\newcommand\ebrw[1]{\ensuremath{\mathsf{\&}\,#1}}
\newcommand\embrw[1]{\ensuremath{\mathsf{\&mut}\,#1}}
\newcommand\eassign[2]{\ensuremath{#1 := #2}}
\newcommand\ederefs[1]{\ensuremath{*^s#1}}
\newcommand\ederefm[1]{\ensuremath{*^m#1}}
\newcommand\ederefb[1]{\ensuremath{*^b#1}}
\newcommand\edrop[1]{\ensuremath{\kw{drop}\,#1}}
\newcommand\emove[1]{\ensuremath{\kw{move}\,#1}}
\newcommand\ecopy[1]{\ensuremath{\kw{copy}\,#1}}
\newcommand\epanic{\kw{panic}}
\newcommand\ereturn{\kw{return}}
\newcommand\eseq[2]{#1;\,#2}
\newcommand\eite[3]{\kw{if}\,#1\,\kw{then}\,#2\,\kw{else}\,#3}
\newcommand\eswitch[3]{\kw{switch}\,#1\;#2\;\mathsf{default}\,#3}
\newcommand\ematch[2]{\kw{match}\,#1\,\kw{ with }\;#2}
\newcommand\enone{\kw{None}}
\newcommand\esome[1]{\kw{Some}\;#1}
\newcommand\krv{\ensuremath{rv}}
\newcommand\kop{\ensuremath{op}}
\newcommand\eloop[1]{\ensuremath{\kw{loop}\,#1}}
\newcommand\ebreak[1]{\ensuremath{\kw{break}\,#1}}
\newcommand\econtinue[1]{\ensuremath{\kw{continue}\,#1}}
\newcommand\kfalse{\mathsf{false}}
\newcommand\ktrue{\mathsf{true}}

\newcommand\emborrow[2]{\ensuremath{\mathsf{borrow}^m\,#1\;#2}}
\newcommand\esborrow[1]{\ensuremath{\mathsf{borrow}^s\,#1}}
\newcommand\eiborrow[1]{\ensuremath{\mathsf{borrow}^r\,#1}}
\let\erborrow\eiborrow
\newcommand\esloan[2]{\ensuremath{\mathsf{loan}^s\,\{#1\}\,#2}}
\newcommand\emloan[1]{\ensuremath{\mathsf{loan}^m\,#1}}
\newcommand\ebox[1]{\ensuremath{\mathsf{Box}\,#1}}

\newcommand\elproj[1]{\ensuremath{\kw{proj}_\mathsf{l}\,#1}}
\newcommand\ebproj[1]{\ensuremath{\kw{proj}_\mathsf{out}\,#1}}
\let\eoproj\ebproj
\newcommand\eiproj[1]{\ensuremath{\kw{proj}_\mathsf{in}\,#1}}
\newcommand\etproj[2]{\ensuremath{\kw{proj}_{#1}\,#2}}

% Ochre commands
% \newcommand{\movearrow}{\ensuremath{\,\stackrel{_(._)}{\Rightarrow}\,}}
\newcommand{\evalarrow}{\ensuremath{\,\Rightarrow\,}}
\newcommand{\ewritearrow}{\ensuremath{\,\Leftarrow\,}}
\newcommand{\movearrow}{\ensuremath{\,\stackrel{_(._)}{\Rightarrow}\,}}
\newcommand{\movearrowconc}{\ensuremath{\,{\Rightarrow}\,}}
\newcommand{\movearrowabs}{\ensuremath{\,\stackrel{.}{\Rightarrow}\,}}
\newcommand{\writearrow}{\ensuremath{\,\stackrel{_(._)}{\Leftarrow}\,}}
\newcommand{\writearrowconc}{\ensuremath{\,{\Leftarrow}\,}}
\newcommand{\writearrowabs}{\ensuremath{\,\stackrel{.}{\Leftarrow}\,}}
\newcommand{\readarrow}{\ensuremath{\,\stackrel{_(._)}{\rightarrow}\,}}
\newcommand{\readarrowconc}{\ensuremath{\,\rightarrow\,}}
\newcommand{\readarrowabs}{\ensuremath{\,\stackrel{.}{\rightarrow}\,}}
\newcommand{\narrowarrow}{\ensuremath{\,\stackrel{_(._)}{\leftarrow}\,}}
\newcommand{\narrowarrowconc}{\ensuremath{\,\leftarrow\,}}
\newcommand{\narrowarrowabs}{\ensuremath{\,\stackrel{.}{\leftarrow}\,}}
\newcommand{\erasedreadarrow}{\ensuremath{\,\dot{\rightsquigarrow}\,}}
\newcommand{\erasedwritearrow}{\ensuremath{\,\dot{\leftsquigarrow}\,}}
\newcommand{\maxarrow}[1]{\ensuremath{\underset{^\kw{max}}{#1}}}
\newcommand{\rearrangearrow}{\ensuremath{\hookrightarrow}}
\newcommand{\mono}[1]{\ensuremath{\text{\texttt{#1}}}}
\newcommand{\loan}[3]{\ensuremath{\kw{loan}^{#1}\,#2\,#3}}
\newcommand{\loans}[2]{\loan{\kw{s}}{#1}{#2}}
\newcommand{\loanm}[1]{\ensuremath{\kw{loan}^m\,#1}}
\newcommand{\loansm}[2]{\loan{\kw{s}|\kw{m}}{#1}{(#2)}}
\newcommand{\borrow}[3]{\ensuremath{\kw{borrow}^{#1}\,#2\,#3}}
\newcommand{\borrows}[2]{\borrow{\kw{s}}{#1}{#2}}
\newcommand{\borrowm}[2]{\borrow{\kw{m}}{#1}{#2}}
\newcommand{\borrowsm}[2]{\borrow{\kw{s}|\kw{m}}{#1}{#2}}
\newcommand{\absenv}[1]{\ensuremath{\{ #1 \}}}
\newcommand{\absmap}[2]{\ensuremath{#1 \, \mapsto \, #2}}
\newcommand{\absmapm}[2]{\absmap{\mono{#1}}{#2}}
\newcommand{\atom}[1]{\ensuremath{\text{'#1}}}
\newcommand{\subtype}{\sqsubseteq}
\newcommand{\typeunion}{\sqcup}
\newcommand{\typeinter}{\sqcap}
\newcommand{\odef}[3][fig:lookuptable]{\ensuremath{\hyperref[#1]{\langle \textit{def.} #2 \textit{for} \, #3 \rangle}}}
\newcommand{\allarrows}{\ensuremath{ \erasedreadarrow, \readarrow, \movearrow, \writearrow, \narrowarrow, \erasedwritearrow }}
\newcommand{\abstractarrows}{\ensuremath{ \erasedreadarrow, \readarrowabs, \movearrowabs, \writearrowabs, \narrowarrowabs, \erasedwritearrow }}
\newcommand{\concreadarrows}{\ensuremath{ \readarrowconc, \movearrowconc }}
\newcommand{\concwritearrows}{\ensuremath{ \writearrowconc, \narrowarrowconc }}
\newcommand{\concarrows}{\ensuremath{ \concreadarrows , \concwritearrows }}
\newcommand{\readarrows}{\ensuremath{ \readarrow, \movearrow }}
\newcommand{\writearrows}{\ensuremath{ \writearrow, \narrowarrow }}
\newcommand{\drop}[1]{\ensuremath{ \kw{drop} \, #1 }}
\newcommand{\concrete}[1]{\ensuremath{ \kw{concrete} \, #1 }}
\newcommand{\oabstract}[1]{\ensuremath{ \kw{abstract} \, #1 }}
\newcommand{\runtime}[1]{\kw{runtime} \, #1}
\newcommand{\comptime}[1]{\kw{comptime} \, #1}
\newcommand{\deref}[1]{\ensuremath{ \mono{*}#1 }}
\newcommand{\oref}[1]{\ensuremath{ \mono{\&}#1 }}
\newcommand{\mutref}[1]{\ensuremath{ \mono{\&\kw{mut}} \, #1 }}
\newcommand{\pleft}[1]{\ensuremath{ #1\mono{.0} }}
\newcommand{\pright}[1]{\ensuremath{ #1\mono{.1} }}
\newcommand{\size}[1]{\kw{size} \, #1}
\newcommand{\oread}[1]{\ensuremath{\kw{read} \, #1}}
\newcommand{\owrite}[1]{\ensuremath{\kw{write} \, #1}}
% \newcommand{\oif}[2]{\ensuremath{\text{if   } #1 \text{   then   } #2}}
\newcommand{\oif}[2]{\ensuremath{#2 \, \, (\text{if    } #1)}}
\newcommand{\oimplies}[2]{\ensuremath{#1 \text{   implies   } #2}}
\newcommand{\oand}[2]{\ensuremath{#1 \text{   and   } #2}}
\newcommand{\oleft}[1]{\ensuremath{\kw{left} \, #1}}
\newcommand{\oright}[1]{\ensuremath{\kw{right} \, #1}}
\newcommand{\oflip}[1]{\ensuremath{(\kw{flip} \, #1)}}
\newcommand{\otab}{\hspace{5ex}}
\newcommand{\abs}[1]{\ensuremath{\, \dot{#1} \,}}

\makeatletter % allow us to mention @-commands
\def\arcr{\@arraycr}
\makeatother

\newcommand{\mcrot}[4]{\multicolumn{#1}{#2}{\rlap{\rotatebox{#3}{#4}~}}}

\ifshort
\newcommand\myparagraph[1]{\emph{#1}.\ }
\fi
\iflong
\let\myparagraph\paragraph
\fi

\newlength{\characterlength}
\settowidth{\characterlength}{a}
\newcommand\cspace{\hspace{\characterlength}}


\hypersetup{pdftitle={},
  pdfsubject={}, 
  pdfauthor={},
  pdfkeywords={}, 
  pdfstartview=FitH,
  pdfpagemode={UseOutlines},% None, FullScreen, UseOutlines
  bookmarksnumbered=true, bookmarksopen=true, colorlinks,
  citecolor=black,%
    filecolor=black,%
    linkcolor=black,%
    urlcolor=black}

\usepackage[all]{hypcap}

\usepackage{todonotes}


%\usepackage{color}
%\usepackage[tight,ugly]{units}
%\usepackage{float}
%\usepackage[colorinlistoftodos]{todonotes}
% \usepackage{ntheorem}
% \theoremstyle{break}
\newtheorem{lemma}{Lemma}
\newtheorem{theorem}{Theorem}
\newtheorem{remark}{Remark}
\newtheorem{definition}{Definition}
\newtheorem{proof}{Proof}


%%% Default fonts
\renewcommand*{\rmdefault}{bch}
\renewcommand*{\ttdefault}{cmtt}



%%% Default settings (page layout)
% \setlength{\parindent}{0em}  % indentation of paragraph

% \setlength{\parindent}{0em}  % indentation of paragraph

\setlength{\parskip}{\baselineskip}

\setlength{\headheight}{14.5pt}
\pagestyle{fancy}
\renewcommand{\chaptermark}[1]{\markboth{\chaptername\ \thechapter.\ #1}{}} 
%\fancyhead[RO]{\sffamily \textbf{\thepage}} %Page no.in the right on even pages
%\fancyhead[LE]{\sffamily \textbf{\thepage}} %Page no. in the left on odd pages

\fancyfoot[ER,OL]{\thepage}%Page no. in the left on
                                %odd pages and on right on even pages
\fancyfoot[OC,EC]{\sffamily }
\renewcommand{\headrulewidth}{0.1pt}
\renewcommand{\footrulewidth}{0.1pt}
\captionsetup{margin=10pt,font=small,labelfont=bf}


%--- chapter heading

\def\@makechapterhead#1{%
  \vspace*{10\p@}%
  {\parindent \z@ \raggedright \sffamily
    \interlinepenalty\@M
    \Huge\bfseries \thechapter \space\space #1\par\nobreak
    \vskip 30\p@
  }}

%--- chapter heading

\def\@makechapterhead#1{%
  \vspace*{10\p@}%
  {\parindent \z@ \raggedright \sffamily
        %{\Large \MakeUppercase{\@chapapp} \space \thechapter}
        %\\
        %\hrulefill
        %\par\nobreak
        %\vskip 10\p@
    \interlinepenalty\@M
    \Huge\bfseries \thechapter \space\space #1\par\nobreak
    \vskip 30\p@
  }}

%---chapter heading for \chapter*  
\def\@makeschapterhead#1{%
  \vspace*{10\p@}%
  {\parindent \z@ \raggedright
    \sffamily
    \interlinepenalty\@M
    \Huge \bfseries  #1\par\nobreak
    \vskip 30\p@
  }}	
\allowdisplaybreaks
